%% brief-template-latex
%%
%% Latex source files for customized letters
%% 
%% Copyright 2010,2017 Martin Michel
%%
%% The layout of this letter is CC-BY licensed, see
%% https://creativecommons.org/licenses/by/3.0/
%
% This work consists of the files :
% brief.tex
% fonts/fontselect.lco
% macros/custommargins.lco
% macros/dateup.lco
% macros/footseperated.lco
% macros/fromaddressnew.lco
% macros/nowindow.lco
% macros/textwidth.lco
%
% The template uses the KOMA-Script class 'scrlttr2' (2010/06/17
% v3.06) 
% Please visit 
%   https://sourceforge.net/projects/koma-script/ 
% for download of latest version
% 
% Download or replace the appropriate fonts for customization

% Check if the following is required on your Latex system
\XeTeXinputencoding cp-1252

% the key/value pairs are self-explanatory, change or comment them out
% as appropriate
\documentclass[
 fontsize=12pt,         
 paper=a4,             
 DIV=14,              
 BCOR=5mm,              
 pagenumber=footcenter,
 parskip=half*,       
 german,              
 raggedright
 ]{scrlttr2}        

\KOMAoptions{
 fromalign=left,
 fromrule=afteraddress,
 fromphone=false,
 fromemail=false,
 fromurl=false,
 foldmarks=false,
 backaddress=true,
 addrfield=true,
 %footsepline=true,
 %headsepline=true
 enlargefirstpage=true
}

% Please change language/localization settings if necessary
\usepackage{fontspec}
\usepackage{xunicode}
%\usepackage{xltxtra}
\usepackage[ngerman]{babel}     
%\usepackage[T1]{fontenc}
%\usepackage[latin1]{inputenc}


\LoadLetterOption{DIN}
\firsthead{
   \flushright
         \fontsize{20}{20}
         \scshape
         %\headerfont      % Uncomment to use specified header font
         \usekomavar{fromname}\\[5mm]
   %\hrule  height  1pt   % Uncomment to add a rule after your name
}

% Activate and edit this macro to change fonts or install the used ones (TTF):
%\LoadLetterOption{fonts/fontselect}   

% Enter your address / contact data here:
\LoadLetterOption{macros/fromaddressnew}

\LoadLetterOption{macros/custommargins}
%\LoadLetterOption{macros/nowindow}
%\LoadLetterOption{macros/dateup}
\LoadLetterOption{macros/footseperated}   % Change language 
\renewcommand*{\raggedsignature}{\raggedright}

% Write the letter ...
\begin{document}

\begin{letter}{Empf�nger}
\setkomavar{toname}{DHL \\ Adrian Fashingbauer}
\setkomavar{toaddress}{Hauptwache 364 \\ 39396 Bretten}
\setkomavar{place}{Wei�enburg}
%\setkomavar{title}{}
\setkomavar{subject}{Man muss die Tatsachen kennen, bevor man sie verdrehen kann. \\ Kundennummer 15650}
\setkomavar{signature}{Waldi Schmid}

\opening{Sehr geehrter Herr Adrian Fashingbauer,}

\begin{flushleft}
Damit das Layout nun nicht nackt im Raume steht und sich klein und leer vorkommt, springe ich ein: der Blindtext. Und weil Sie nun schon die G�te haben, mich ein paar weitere S�tze lang zu begleiten, m�chte ich diese Gelegenheit nutzen, Ihnen nicht nur als L�ckenf�ller zu dienen, sondern auf etwas hinzuweisen, das es ebenso verdient wahrgenommen zu werden: Webstandards n�mlich.

Gehetzt sah er sich um. Und es war ihnen wie eine Best�tigung ihrer neuen Tr�ume und guten Absichten, als am Ziele ihrer Fahrt die Tochter als erste sich erhob und ihren jungen K�rper dehnte. Das Layout ist fertig, der Text l�sst auf sich warten. Eines Tages aber beschlo� eine kleine Zeile Blindtext, ihr Name war Lorem Ipsum, hinaus zu gehen in die weite Grammatik. Wer w�rde ihm schon folgen, sp�t in der Nacht und dazu noch in dieser engen Gasse mitten im �bel beleumundeten Hafenviertel?
\end{flushleft}

\closing{Mit freundlichen Gr��en}

\removelastskip

\end{letter}

\end{document}
